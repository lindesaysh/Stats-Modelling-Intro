%============================================================================%
% St Andrews bookdown template 
% LaTeX preamble file
%============================================================================%
\renewcommand{\familydefault}{\sfdefault}

%============================================================================%
% Generic packages
%============================================================================%
\usepackage{booktabs}
\usepackage{longtable}
\usepackage{amsthm}

%============================================================================%
% Change Bibliography to References to match gitbook output
%============================================================================%
\renewcommand{\bibname}{References}

%============================================================================%
% Place all figures and tables after code chunk in PDF output
%============================================================================%
\usepackage{float}
\floatplacement{figure}{H}
\floatplacement{table}{H}

%============================================================================%
% Make index so we can print it later (see: afterbody.tex)
%============================================================================%
\usepackage{makeidx}
\makeindex

%============================================================================%
% Copied from Yihui's repo 
% https://github.com/rstudio/bookdown/tree/master/inst/examples
% First part to change spacing, second to cater for xetex latex engine
%============================================================================%
\makeatletter
\def\thm@space@setup{%
  \thm@preskip=8pt plus 2pt minus 4pt
  \thm@postskip=\thm@preskip
}
\makeatother

\ifxetex
  \usepackage{letltxmacro}
  % \setlength{\XeTeXLinkMargin}{1pt}
  \LetLtxMacro\SavedIncludeGraphics\includegraphics
  \def\includegraphics#1#{% #1 catches optional stuff (star/opt. arg.)
    \IncludeGraphicsAux{#1}%
  }%
  \newcommand*{\IncludeGraphicsAux}[2]{%
    \XeTeXLinkBox{%
      \SavedIncludeGraphics#1{#2}%
    }%
  }%
\fi

%============================================================================%
% LaTeX macro to create task boxes
%============================================================================%
\usepackage{tcolorbox}
\tcbuselibrary{breakable}
\definecolor{taskCol}{HTML}{00539b}
\definecolor{taskCol1}{HTML}{007870}
\tcbset{colback=white,colframe=taskCol,arc=0mm}

% Trick to fool markdown into compiling
\newcommand{\bblockT}[2][Task]{\begin{tcolorbox}[title = #1 #2, parbox = false]}
\newcommand{\eblockT}{\end{tcolorbox}}
\newcommand{\bblockS}[2][Solution]{\begin{tcolorbox}[title = #1 #2, colframe=taskCol1, breakable, parbox = false]}
\newcommand{\eblockS}{\end{tcolorbox}}

% Add tabbed solutions environment
\newcommand{\bmp}{\begin{minipage}[c]{0.5\textwidth}}
\newcommand{\emp}{\end{minipage}}
\newcommand{\bblockST}[1]{\begin{tcolorbox}[title = #1, colframe=taskCol1, breakable, parbox = false]}
\newcommand{\eblockST}{\end{tcolorbox}}

% Set solution button link
\usepackage{tikz}

\newcommand{\buttonT}[1]{
    \begin{tikzpicture}
    \node[
        inner sep=5pt,
        draw=taskCol,
        fill=taskCol,
        rounded corners=2pt,
        text=white
    ] (c1) {#1};
    \end{tikzpicture}
}

\newcommand{\buttonS}[1]{
    \begin{tikzpicture}
    \node[
        inner sep=5pt,
        draw=taskCol1,
        fill=taskCol1,
        rounded corners=2pt,
        text=white
    ] (c1) {#1};
    \end{tikzpicture}
}
\newcommand{\colpageref}[1]{\hypersetup{linkcolor=white}\pageref{#1}}